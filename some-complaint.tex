\providecommand{\documentclassflag}{}
\documentclass[12pt,\documentclassflag]{complaint} 

\usepackage[margin=1in]{geometry}
\usepackage{microtype,newcent}
\usepackage{setspace,xcolor}
%\usepackage[hyperindex=false,linkbordercolor=white]{hyperref}

\makeandletter
\citecase[Greeting Card Publishers]{National Assoc. of Greeting Card Publishers v. United States Postal Service, 462 U.S. 810 (1983)}
\citecase{Wiener v. United States, 357 U.S. 349 (1958)}
\citecase{Humphrey's Ex'r v. United States, 295 U.S. 602 (1935)}
\citecase{Marbury v. Madison, 5 U.S. 137 (1803)}
\citecase{National Treasury Employees Union v. Nixon, 492 F.2d 587 (D.C. Cir. 1974)}
\citecase{Mississippi v. Johnson, 71 U.S. 475 (1866)}
\citecase[Holiday Tours]{Washington Metropolitan Area Transit Commission v. Holiday Tours, Inc., 559 F.2d 841 (D.C. Cir. 1977)}
\citecase[Steel Seizure Case]{Youngstown Sheet and Tube Co. v. Sawyer, 343 U.S. 579 (1952)}
\citecase[McLennan]{United States ex rel McLennan v. Wilbur, 283 U.S. 414 (1931)}

\statutesindextitle{Iowa Court Rules}
\newstatute{Rule 51}{}
%\newstatute{39 U.S.C.}{(1988)}
%\newstatute{Fed. R. Civ. P.}{}
%\SetIndexName{39 U.S.C.}{39@Title 39, United States Code !}
\%SetIndexName{Fed. R. Civ. P.}{Federal Rules of Civil Procedure !Rule }
\SetIndexName{Rule 51}{Iowa Code of Judicial Conduct !}

\newmisc{Black's Law Dictionary}{Black's Law Dictionary (8th ed. 2004)}

\newmisc{Elcano Decl.}{January 6, 1993 Declaration of Mary S. Elcano\pin{ }{}}
\SetIndexType{Elcano Decl.}{} %Suppress from the Table of Authorities
 
\plaintiffs{\begin{center}In the Matter of the Conduct of
Judge Jane Doe of the 5th Judicial District\end{center}}
\firstpartytitle{}
%\defendant{George Bush, in his official capacity as President of the United States} 
\court{Before the Commission on Judicial Qualifications of the State of Iowa}
\
\actionnumber{\begin{center}Complaint to the Commission on\newline Judicial Qualifications of\newline the State of Iowa\end{center}}


\parindent=2em
\setlength{\parskip}{1.25ex plus 2ex minus .5ex} 
\doublespacing

\begin{document}

%================ MOTION =======================
\makecaption

\section*{Facts and Procedural History}
\thispagestyle{empty}
\begin{enumerate}

\item On such and such day at such and such time such and such happened







\section{Failure to Comply with the Law} 
 \See \pincite{Rule 51}{\S 1.1 COMPLIANCE WITH THE LAW}.  ``A judge shall comply with the law, including the Iowa Code of Judicial Conduct.''

\item Facts 3,4,4,5  constitute a failure to comply with Iowa Code \S XXX  


%Rule 51:1.1: COMPLIANCE WITH THE LAW
%A judge shall comply with the law,* including the Iowa Code of Judicial Conduct.
 



Rule 51:1.2: PROMOTING CONFIDENCE IN THE JUDICIARY
A judge shall act at all times in a manner that promotes public confidence in the independence, integrity, and impartiality of the judiciary and shall avoid impropriety and the appearance of impropriety.

Rule 51:2.2: IMPARTIALITY AND FAIRNESS
A judge shall uphold and apply the law,* and shall perform all duties of judicial office fairly and impartially.

Rule 51:2.3: BIAS, PREJUDICE, AND HARASSMENT
(A) A judge shall perform the duties of judicial office, including administrative duties, without bias or prejudice.
(B) A judge shall not, in the performance of judicial duties, by words or conduct manifest bias or prejudice or engage in harassment, including but not limited to bias, prejudice, or harassment based upon race, sex, gender, religion, national origin, ethnicity, disability, age, sexual orientation, marital status, socioeconomic status, or political affiliation, and shall not permit court staff, court officials, or others subject to the judge?s direction and control to do so.
(C) A judge shall require lawyers in proceedings before the court to refrain from manifesting bias or prejudice or engaging in harassment, based upon attributes including but not limited to race, sex, gender, religion, national origin, ethnicity, disability, age, sexual orientation, marital status, socioeconomic status, or political affiliation, against parties, witnesses, lawyers, or others.
(D) The restrictions of paragraphs (B) and (C) do not preclude judges or lawyers from making legitimate reference to the listed factors, or similar factors, when they are relevant to an issue in a proceeding.


Rule 51:2.4: EXTERNAL INFLUENCES ON JUDICIAL CONDUCT
(A) A judge shall not be swayed by public clamor or fear of criticism.
(B)A judge shall not permit family, social, political, financial, or other interests or relationships to influence the judge?s judicial conduct or judgment.
(C) A judge shall not convey or permit others to convey the impression that any person or organization is in a position to influence the judge.


Rule 51:2.5: COMPETENCE, DILIGENCE, AND COOPERATION
(A) A judge shall perform judicial and administrative duties competently and diligently.
(B) A judge shall cooperate with other judges and court officials in the administration of court business.


Rule 51:2.6: ENSURING THE RIGHT TO BE HEARD
(A) A judge shall accord to every person who has a legal interest in a proceeding, or that person?s lawyer, the right to he heard according to law.
(B) A judge may encourage parties to a proceeding and their lawyers to settle matters in dispute but shall not act in a manner that coerces any party into settlement.


Rule 51:2.7: RESPONSIBILITY TO DECIDE
A judge shall hear and decide matters assigned to the judge, except when disqualification is required by rule 2.11 or other law.


Rule 51:2.11: DISQUALIFICATION
(A) A judge shall disqualify himself or herself in any proceeding in which the judge?s impartiality might reasonably be questioned, including but not limited to the following circumstances:
(1) The judge has a personal bias or prejudice concerning a party or a party?s lawyer, or personal knowledge of facts that are in dispute in the proceeding.
(2) The judge knows* that the judge, the judge?s spouse or domestic partner,* or a person within the third degree of relationship* to either of them, or the spouse or domestic partner of such a person is:
(a) a party to the proceeding, or an officer, director, general partner, managing member, or trustee of a party;
(b) acting as a lawyer in the proceeding;
(c) a person who has more than a de minimis interest that could be substantially affected by the proceeding; or
(d) likely to be a material witness in the proceeding.
(3) The judge knows that he or she, individually or as a fiduciary,* or the judge?s spouse, domestic partner, parent, or child, or any other member of the judge?s family residing in the judge?s household,* has an economic interest* in the subject matter in controversy or in a party to the proceeding.
(4) The judge knows or learns by means of disclosures mandated by law* or a timely motion that the judge?s participation in a matter or proceeding would violate due process of law as a result of:
(a) campaign contributions made by donors associated* or affiliated* with a party or counsel appearing before the court, or
(b) independent campaign expenditures by a person* other than a judge?s campaign committee, whose donors to the independent campaign are associated or affiliated with a party or counsel appearing before the court.
(5) The judge, while a judge or a judicial candidate,* has made a public statement, other than in a court proceeding, judicial decision, or opinion, that commits or appears to commit the judge to reach a particular result or rule in a particular way in the proceeding or controversy.
(6) The judge:
(a) served as a lawyer in the matter in controversy or was associated with a lawyer who participated substantially as a lawyer in the matter during such association;
(b) served in governmental employment and in such capacity participated personally and substantially as a lawyer or public official concerning the proceeding, or has publicly expressed in such capacity an opinion concerning the merits of the particular matter in controversy;
(c) was a material witness concerning the matter; or
(d) previously presided as a judge over the matter in another court.
(B) A judge shall keep informed about the judge?s personal and fiduciary economic interests
and make a reasonable effort to keep informed about the personal economic interests of the judge?s spouse or domestic partner and minor children residing in the judge?s household.
(C) A judge subject to disqualification under this rule, other than for bias or prejudice under paragraph (A)(1), may disclose on the record the basis of the judge?s disqualification and may ask the parties and their lawyers to consider, outside the presence of the judge and court personnel, whether to waive disqualification. If, following the disclosure, the parties and lawyers agree, without participation by the judge or court personnel, that the judge should not be disqualified, the judge may participate in the proceeding. The agreement shall be incorporated into the record of the proceeding.

Rule 51:2.14: DISABILITY AND IMPAIRMENT
A judge having a reasonable belief that the performance of a lawyer or another judge is impaired by drugs or alcohol, or by a mental, emotional, or physical condition, shall take appropriate action, which may include a confidential referral to a lawyer or judicial assistance program.

Rule 51:2.15: RESPONDING TO JUDICIAL AND LAWYER MISCONDUCT
(A) A judge having knowledge* that another judge has committed a violation of the Iowa Code of Judicial Conduct that raises a substantial question regarding the judge?s honesty, trustworthiness, or fitness as a judge in other respects shall inform the appropriate authority.
(B) A judge having knowledge that a lawyer has committed a violation of the Iowa Rules of Professional Conduct that raises a substantial question regarding the lawyer?s honesty, trustworthiness, or fitness as a lawyer in other respects shall inform the appropriate authority.
(C) A judge who receives information indicating a substantial likelihood that another judge has committed a violation of this Code shall take appropriate action.
(D) A judge who receives information indicating a substantial likelihood a lawyer has committed a violation of this Code shall take appropriate action.
(E) This rule does not require disclosure of information gained by a judge while participating in an approved judges or lawyers assistance program.


Rule 51:2.16: COOPERATION WITH DISCIPLINARY AUTHORITIES
(A) A judge shall cooperate and be candid and honest with judicial and lawyer disciplinary agencies.
(B) A judge shall not retaliate, directly or indirectly, against a person known* or suspected to have assisted or cooperated with an investigation of a judge or a lawyer.

\end{enumerate}
Date: August 21, 2019

\begin{rightbox}
Respectfully Submitted,\\
~\\
John Q. Lawyer\\
127 Wall Street\\
New Haven, CT 06511\\
\end{rightbox}
\newpage

%============= PROPOSED ORDER ==================
%\makecaption
%\section*{Order}
%\thispagestyle{empty}
%
%AND NOW, this \underline{~~~~~} day of \underline{~~~~~~~~~~}, 1993, upon consideration of Plaintiffs' Motion for a Temporary Restraining Order it is hereby
%
%ORDERED that Defendant George Bush is enjoined from removing Bert H. Mackie, Norma Pace, John N. Griesemer, Crocker Nevin, Tirso Del Junco, Marvin T. Runyon and Michael S. Coughlin from their positions as members of the Board of Governors of the Postal Service.
%
%\begin{rightbox}
%~\\
%~\\
%\underline{~~~~~~~~~~~~~~~~~~~~~~~~~~~~~~~~~~~~~~~~~~~}\\
%United States District Court Judge
%\end{rightbox}
%\newpage
%
%================ BRIEF =======================
\makecaption

\section*{Plaintiffs' Statement of Points and Authorities In Support of the \\Motion for a Temporary Restraining Order}
\pagenumbering{roman}
\pagestyle{romanparen}
\thispagestyle{empty}
\newpage

\section*{Table of Contents}
\singlespacing
\tableofcontents \newpage

\begingroup
%Uncomment the below to replace the section symbol with the word "Section"
%However, the default makeindex does not properly sort these numerically
%\protected\def\S{\ }
%\def\Ss{Section }
\tableofauthorities \newpage
\endgroup

\pagestyle{plain}
\pagenumbering{arabic}

\section{Introduction}
\doublespacing

%\tracingmacros=2
%\S
%\show\S

Plaintiffs request a temporary restraining order to enjoin the President from illegally removing them without cause as members of the Board of Governors of the Postal Service.  The Postal Service was established by Congress as an agency independent of political authority, and outside the direction of the President.  It is directed by a Board of Governors that---while appointed by the President and under the advice and consent of the Senate---are chosen solely for their economic and managerial skills, and whose task is to run the Postal Service in a professional, business-like manner.  An inevitable result of their independence is that they will at times have different views than those of the President, and in fact, they now take a different litigation stance than does the President on a rate-setting matter.  The President has threatened to retaliate against the Governors for their independent stance by illegally removing them from office.  

Were the President allowed to persist in this course of action, the independence of future Governors would be permanently curtailed, and the will of Congress in establishing this independent agency would be thwarted.  To prevent this irreparable injury, the Governors request this Court to enter a temporary restraining order enjoining the President from removing the Governors.  An ex parte order is appropriate in this case because the President, were he to be informed of this action in advance of a hearing, is liable to remove the Governors immediately. Therefore, to preserve the status quo so that this matter may be properly adjudicated, a temporary restraining order is needed.  

It is no matter that the President of the United States is the subject of the requested injunction.  The President is a public officer like any other, and subject to the laws.  In order to see that the laws of this country are properly respected, this Court has the authority to enjoin the President from violating them.   
 
\section{Statement of Facts}

\subsection{Statutory Framework}

The current organization of the Postal Service results from Congress's desire to ``get the politics out of the Post Office.'' \citetext{\pincite{Greeting Card Publishers}{822} (internal quotation marks omitted).} With the enactment of the 1970 Postal Reorganization Act 
\citeclause{\pincite{39 U.S.C.}{\S\S 101 \emph{et seq.}}} Congress ``divested itself of the control it theretofore had exercised over the setting of postal rates and fees\IfLawReview{.}{,}'' \pincite[i]{Greeting Card Publishers}{813}. In relinquishing this control, Congress was ``disturbed about the influence of lobbyists on Congress' discretionary ratemaking.'' \pincite{Greeting Card Publishers}{822}. Therefore, it ``removed the rate-setting function from the political arena by removing postal funding from the budgetary process \ldots and by removing the Postal Service's principal officers from the President's direct control.'' \pincite{Greeting Card Publishers}{822}. 

The Plaintiffs are Governors of the United States Postal Service, an independent agency created by Congress \citeclause{\see \pincite{39 U.S.C.}{\S 201}} under the direction and authority of the Board of Governors \citeclause{\see \pincite{39 U.S.C.}{\S 202(a)(1)}}.  The Board consists of 11 members, nine of which are appointed by the President with the advice and consent of the Senate. \See \pincite{39 U.S.C.}{\S 202(a)(1)}. All Governors shall be ``chosen solely on the basis of their experience in the field of public service, law or accounting or on their demonstrated ability in managing organizations or corporations \ldots of substantial size.'' \pincite{39 U.S.C.}{\S 202(a)(1)}.  Significantly, the ``[g]overnors \ldots may be removed only for cause.'' \pincite{39 U.S.C.}{\S 202(a)(1)}.  

Simultaneously, Congress created the independent Postal Rate Commission. \See \pincite{39 U.S.C.}{\S 3601}.   If Board of Governors decides that changes in postal rates are necessary, they must request a change from the Rate Commission. \See \pincite{39 U.S.C.}{\S 3601}. It is the Rate Commission that has the power to specify a new rate schedule, after holding hearings. \See \pincite{39 U.S.C.}{\S 3624}.  Postal rates and fees shall provide sufficient revenues so that estimated income will approximately equal estimated expenses. \See \pincite{39 U.S.C.}{\S 3621}. The Governors may then approve the new schedule, allow it under protest, reject, or (when unanimous and under limited circumstances) modify the recommendation of the Rate Commission. \See \pincite{39 U.S.C.}{\S 3625}. The public may appeal in any United States Court of Appeals any decision of the Governors to approve, allow under protest, or modify the recommended decision of the Postal Rate Commission. \See \pincite{39 U.S.C.}{\S 3628}. The Governors may also seek judicial review of the Rate Commission's decision by allowing it under protest. \See \pincite{39 U.S.C.}{\S 3625(c)}. 

The conflict between the Governors and the President centers on the proper legal representative of the Governors in cases arising under Sections 3625(c) and 3628.  In these and other legal matters, Postal Service is ordinarily represented by counsel furnished by the Department of Justice.  \See \pincite{39 U.S.C.}{\S 409(d)}.  However, ``with the prior consent of the Attorney General the Postal Service may employ attorneys by contract or otherwise to conduct litigation brought by or against the Postal Service.''  \pincite{39 U.S.C.}{\S 409(d)}.  

\subsection{Factual Case History}
On March 6, 1990, in accordance with its authority under 39 U.S.C. \S\S 3622 and 3623, the Board of Governors requested a recommendation from the Postal Rate Commission on the setting of new rates and fees.  \citetext{Compl.\ \P 29.}  The Commission responded with a recommendation on January 4, 1991.  \citetext{Compl.\ \P 30.}  In response, under \S 3625, the Board issued three unanimous decisions to: (1) allow the rate schedule under protest and return it for reconsideration, (2)  allow under protest and to seek judicial review of the Commission's recommendation of a so-called ``Public's Automation Rate'' (PAR), and (3) reject several classification changes. \See \pincite{Elcano Decl.}{\P4; Attachment A}. By a letter dated January 23, 1991, the Postal Service requested the Attorney General to consent under 39 U.S.C. \S 409(d) to the Postal Service's self-representation in the PAR appeal, and in actions by the public that were expected to follow the Board's decisions. \See \pincite{Elcano Decl.}{\unskip, Attachment A}. Ten separate actions were filed.  \See \pincite{Elcano Decl.}{\P 6}.

At first, the Postal Service received no formal response.  \See \pincite{Elcano Decl.}{\P 5}.  In a meeting on February 11, 1991, the Department of Justice (DOJ) was informed that an immediate filing was required to preserve the Governors' decision to seek judicial review of the PAR decision, but the DOJ did nothing.  \See \id[\P 7]. The Postal Service was forced to file itself, without approval---or it would have waived its rights to judicial review. \See \id[\P 8].  The DOJ orally noted that it had not approved, but took no further action. \See \id.  If the Postal Service had not prepared and filed briefs on its own behalf, the Governors would have been left totally unrepresented in their appeal. \See \id[\P 11]. 

The Department of Justice informed the Postal Service on September 29, 1992 that it was denying the Postal Service's request for self-representation with respect to the private suits, as requested by the Rate Commission. \See \id[\P 12, Attachment B].  However, the DOJ offered to prepare a joint brief representing the views held by the Governors and the Rate Commission---despite the fact that the Rate Commission was not a party to any docket other than that in which it was the defendant (the PAR matter). \See \id[\P 13].  Irreconcilable differences arose between the positions held by the Governors and those of the DOJ. \See \id[\P 17].  The Postal Service thus sought leave from the Court to represent itself. \See \id[\P 18].  This prompted the Counsel to the President, C. Boyden Gray, to call a meeting with the Postmaster General, the Chairman of the Rate Commission, and several officials from DOJ on December 8, 1992. \See \id[\P 20].  The Counsel to the President stated that the Administration intended to supplant the judicial oversight provided by the Postal Reorganization Act with a alternative procedure for dispute resolution within the Executive Branch. \See \id[\P 22, Attachment J].

On December 11, 1992, the President issued a memorandum directing that the Board of Governors cooperate with the withdrawal of their filings in the lawsuits. \See \id[\P 23, Attachment K]. The Governors informed the President on December 16, 1992, that they would consider how to respond at their next meeting, scheduled for January 4, 1993.  \See \id[\P 24, Attachment M].  On January 4, 1993, the President sent each of the Governors a letter expressing his view that the Postal Service was in violation of \S409(d), and ``in order to obtain compliance with the statutes and my directive enforcing them, I will if necessary exercise my authority to remove Governors of the Postal Service.'' \See \pincite{Elcano Decl.}{\P 27, Attachment P}.

\section{Argument}

A temporary restraining order may be granted only when the plaintiff demonstrates (1) a substantial likelihood of success on the merits; (2) that irreparable injury will result in the absence of the requested relief; (3) that no other parties will be harmed if temporary relief is granted; and (4) that the public interest favors entry of a temporary restraining order. \pincite{Holiday Tours}{843}.  All four criteria are met in this case.

\subsection{The Governors Have Substantial Likelihood of Success On The Merits}

The Governors seek declaratory judgment that the President cannot remove them from office without cause, and seek a permanent injunction prohibiting him from the same.  Because it is well-settled law that the Courts have the power to enjoin the president, that the Governors of independent agencies cannot be removed without cause, and that a policy disagreement does not give the President cause to remove the Governors, they have a substantial likelihood of success on the merits. 

\subsubsection{This Court has the Authority to Enjoin the President}

Courts have the power to order executive officials to perform or refrain from activities as required by law.   To issue such an order, the requested relief must be ``a precise course accurately marked out by law, and [one] to be strictly pursued.'' \citetext{\pincite{Marbury}{157}.   \Seealso \pincite{McLennan}{420} (``the law must not only authorize the demanded action, but require it.'').}   Such non-discretionary functions of officials are contrasted with their political functions, which involve the exercise of discretion.   

In contrast to non-discretionary functions, an official cannot be enjoined from an exercise of his or her political authority.  \citetext{\See \pincite{Mississippi}{499} (``general principles [] forbid judicial interference with the exercise of Executive discretion'').}  In \emph{Mississippi}, the State sought to enjoin President Johnson from enforcing an ostensibly unconstitutional law.  \pincite{Mississippi}{498}. 
 
However, the Court found that the law, which required the President to make military decisions in the occupation of the post--Civil War South, involved the exercise of significant discretion. \pincite{Mississippi}{499}. Therefore, the injunction was improper. \citetext{\See \id. at 500.} 

\emph{Mississippi} does not hold that the President may not be enjoined.  Rather, the holding is limited to the proposition he that cannot be ``restrained by injunction from carrying into effect an act of Congress alleged to be unconstitutional.'' \pincite{Mississippi}{498}.  The Court expressed no opinion ``whether, in any case, the President of the United States may be required \ldots to perform a purely ministerial act.'' \pincite{Mississippi}{498}. 

Since \emph{Marbury}, an ``unbroken line of authority'' holds that courts have the power to issue mandatory orders to federal officers other than the President. \pincite{National Treasury Employees Union}{602}.  Subsequently, however, to ensure the separation of powers, this power of the Judiciary has been extended to enjoining the President directly.  In the \emph{Steel Seizure Case}, the court was asked to decide ``whether the President was acting within his constitutional power when he issued an order directing the Secretary of Commerce to take possession of and operate most of the Nation's steel mills.'' \pincite{Steel Seizure Case}{582}.  The plaintiffs requested an injunction against the President's order, which was granted by the district court \citeclause{\see \id. at 583} and affirmed by the Supreme Court \citeclause{\see \id. at 589}.  While the Secretary of Commerce was the direct target of the injunction \citeclause{\see \id. at 582} the Court left little doubt that the President's authority, and not that of the Secretary, was called into question.  \citetext{\See \id. at 584 (``is the seizure  order within the constitutional power of the President?''); \pincite{National Treasury Employees Union}{611} (``There is not the slightest hint in any of the Youngstown opinions that the case would have been viewed differently if President Truman rather than Secretary Sawyer had been the named party.'').}

This Circuit has ruled that when no lesser official is a suitable target for an injunction, as the Secretary of Commerce was in the \emph{Steel Seizure Case}, the President himself may be enjoined. \See \cite[s]{National Treasury Employees Union}.  Without such a power, the President could evade judicial oversight for actions of the Executive branch by exercising responsibility directly, rather than through delegation of authority.   \See \pincite{National Treasury Employees Union}{613}.  It is not by the office of the person to whom the order is directed, but the nature of the thing to be done that the propriety of issuing an order is to be determined. \pincite{Marbury}{166}. When the President acts or threatens to act in violation of a non-discretionary duty, the Courts have the power to enjoin him.

\subsubsection{Congress May Establish Independent Agencies, Whose Officers May Not Be Removed At The President's Discretion}

Congress may create agencies that are designed to be independent from ordinary political control.  The officers that direct these agencies may be removed only for cause, such as neglect or malfeasance.  They may not be removed by the President simply because he wants to replace them.  The President's ability to remove these officers at will would eliminate their independence.  \citetext{\Cf \pincite{Humphrey's Ex'r}{629} (``[I]t is quite evident that one who holds his office only during the pleasure of another cannot be depended upon to maintain an attitude of independence against the latter's will.'').} 

The law recognizes a sharp distinction between those officials ``who [are] part of the Executive establishment and [are] thus removable by virtue of the President's constitutional powers, and those \ldots as to whom a power of removal exists only if Congress may said to have conferred it.'' \pincite{Wiener}{352}.  These two categories can be distinguished by the function they perform.  The former category includes offices that exercise delegated executive power.  They can be said to merely exercise the President's own power.  The latter category includes those that call on their officers to exercise independent, apolitical judgment and ``whose tasks require absolute freedom from Executive interference.'' \pincite{Wiener}{353}.  Supreme Court precedent consistently confirms Congress's power to create offices whose occupants may not be removed without cause. \citetext{\Seeeg \cite{Humphrey's Ex'r} (Members of the Federal Trade Commission); \cite{Wiener} (Members of the War Claims Commission).} 

\subsubsection{The Postal Service Is Such An Independent Agency, Therefore Its Governors May Not Be Removed Without Cause} 

 Whether or not an officer is independent of the President depends on Congress's intent in creating the position.   Therefore, to recognize such officials, we must look at the role that Congress intended the officials to fulfill.  Independent agencies are designed to be nonpartisan, apolitical, quasi-legislative rather than purely executive, and ``exercise the trained judgment of a body of experts appointed by law and informed by experience.'' \pincite{Humphrey's Ex'r}{624} (internal quotation marks omitted).  Furthermore, Congress may state explicitly that certain officers may not be removed without cause, or this may be inferred from the language creating the office. For example, a statute providing that an official with a definite term may be removed ``for inefficiency, neglect of duty, or malfeasance'' implies that, absent such conduct, the official may not be removed. \See \pincite{Humphrey's Ex'r}{619}.   

The Governors of the Postal Service are in the class of independent officials that the President may not arbitrarily dismiss.  First, Congress has stated explicitly that ``[g]overnors \ldots may be removed only for cause.'' \pincite{39 U.S.C.}{\S 202(a)(1)}.  Second, independence is inferred from the nature of the Postal Service.  It is nonpartisan.  \See  \pincite{39 U.S.C.}{\S 202(a)(1)} (``not more than 5 of [the Governors] may be adherents of the same political party'').  It is apolitical. \See \pincite{Greeting Card Publishers}{822} (``Congress sought to get politics out of the Post Office'') (internal quotation marks omitted). It exercises a quasi-legislative function---enacting new postal rates---previously reserved to Congress. \See \pincite{Greeting Card Publishers}{813}. And officers are chosen ``solely on the basis of their experience in the field of public service, law or accounting or on their demonstrated ability in managing organizations or corporations.'' \pincite{39 U.S.C.}{\S 202(a)(1)}.  These facts imply that the Postal Service was intended by Congress to be independent---Congress did not relinquish its control only to hand it over to the President.  Therefore its Governors may not be subject to arbitrary Presidential control and removal.

\subsubsection{A Disagreement Over Litigation Is Not Cause For Removal}

The statutory framework makes clear that the Governors will be party to litigation, and have a right that their views be represented.   The Governors are the target of litigation under Section 3628, which provides for judicial oversight of their decision to adopt new rate schedules.  The Governors also initiate litigation under Section 3625(c), in which they appeal decisions of the Rate Commission.  It is clear that Congress envisioned that litigation and judicial review were to be a critical part of the rate-setting process. 

In such litigation, the Governors have a right to representation of their views.  Section 409(d) provides that, ordinarily, the Department of Justice will provide representation.  However, Congress allowed a mechanism by which the Postal Service could represent itself, in such cases where the DOJ was unable to properly represent the views of the Board: the Attorney General may consent to the Postal Service's self-representation. 

However, the Attorney General may not withhold his consent for the purpose of altering the litigation stance of the Governors.  To do so would be to subordinate the formerly independent Postal Service to the will of the President.  Congress created the Postal Service to remove rate-setting from the political sphere.  Because litigation is an integral part of the rate-setting process, were the President to control that litigation, he would have regained control over rate-setting.  Therefore the Postal Service must have the ability to have its own views represented in such litigation if it is to maintain its independence.

A different stance on pending litigation, therefore, is not cause for removal of the Governors.  Rather, such conflict was an inevitable result when Congress created the Postal Service as an independent agency.  Were the President able to remove the Governors on this grounds, the Postal Service would no longer be independent. 

The Postal Service and the President disagree on the interpretation of statutes relating to the Postal Service's right of representation.  Because of his interpretation, the President claims cause to remove the Governors.  \See \pincite{Elcano Decl.}{Attachment P}. However, ``nothing in the Constitution commits to the President the ultimate authority to construe federal statutes.'' \pincite{National Treasury Employees Union}{604}.  Rather, ``it is, emphatically, the province and duty of the judicial department, to say what the law is.'' \pincite{Marbury}{177}.  Thus, when conflicts inevitably arise between the Executive and the independent agencies created by Congress, the Courts must resolve them.   The President may not remove the Governors over this policy disagreement.

\subsection{The Governors Face Immediate and Irreparable Injury If The TRO Is Not Granted}

If the Governors are removed from office before the conclusion of their term, they will be irreparably injured.  An irreparable injury is one ``that cannot be adequately measured or compensated by money.'' \cite{Black's Law Dictionary}.  While part of the damage is monetary---the loss of salary---they also have the right to continue in office for the duration of their term.  This right cannot be measured in monetary amounts. The deprivation of this right is irreparable.

Furthermore, the ex parte restraining order is appropriate, rather than waiting for a full hearing. A temporary restraining order may be granted without written or oral notice to the adverse party or that party's attorney only if immediate and irreparable injury, loss, or damage will result to the applicant before the adverse party can be heard in opposition. \See \pincite{Fed. R. Civ. P.}{65}.  The President has threatened to remove the Governors by today, January 6, 1993.  \See \pincite{Elcano Decl.}{\P 27, Attachment P}. It is reasonable to assume that if the opposing party is notified, they will immediately carry out their threat to remove the Governors.  Therefore, the irreparable injury that this TRO seeks to address is immediate.  This Court should issue the TRO to preserve the issue for trial. 
 
\subsection{No Third Parties Will Be Harmed, And The Public Interest Favors Granting The TRO}

No third parties are directly affected by the requested TRO.  The TRO maintains the status quo, and will not alter the interactions of third parties with the Postal Service.  However, a failure to issue the TRO will cause uncertain leadership in the Postal Service.  The President is likely to exercise his recess power to appoint replacement Governors to the Board.  Such an action would compound the damages suffered by the Plaintiffs, as they would alter the policies that the Plaintiffs had a right to put in place as Governors of the Postal Service.   

In contrast, this Court will cause at worst a minor delay for the Executive Branch by issuing a restraining order now.  The obvious disadvantage of an ex parte TRO is that it restrains the actions of one party before that party has been able to fully adjudicate it's rights, or even be heard in court.  However, such concerns may be mitigated if the period of time before a full hearing are short, especially if the failure to issue the order will cause severe hardship to the other party.  In this case, no factual issues are in dispute between the parties, and no testimony or discovery are required.  As a dispute over a question of law, this matter will be resolved in summary judgment, most probably within several weeks.  If the President were to win at that time, his damage is limited to having suffered a short delay.  However, without the injunction, even if the Governors were to win their case on the merits, they would likely have already suffered the irreparable loss of their position.  Therefore the balance of equities favors granting the TRO.

\section{Conclusion}

For the reasons set forth above, Judge Doe should be referred to the Iowa Supreme Court and Iowa House for disciplinary proceedings.

\noindent Date: January 6, 1993

\begin{rightbox}
Respectfully Submitted,\\
~\\
John Q. Lawyer\\
127 Wall Street\\
New Haven, CT 06511\\
\end{rightbox}

\end{document}


